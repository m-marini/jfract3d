\documentclass{article}
\begin{document}
  \title{Spazio degli stati \thanks{Versione \$Id: Stati.tex,v 1.1 2007/08/31 01:19:50 marco Exp $ $}}
  \author{Marco Marini}
  \maketitle
  \part{Main}
  \section{Base}
  
  Sia $t\vec{V}$ una funzione vettoriale in $t$ con $0\le t\le 1$.
  
  Siano $T_i$ $n$ funzioni di trasformazione di vettori con $i=0 \dots (n-1)$.
  
  Sia $t\varphi=nt-floor(nt)$ funzione di $t$ abbiamo che $0\le t\varphi< 1$.
  
  Sia $t\sigma=floor(nt)$ funzione intera di $t$ abbiamo che $t\sigma=0\dots(n-1)$.
  
  Definiamo
  
  \begin{equation}
    T^m_t=\prod_{i=0}^{m-1}T_{t\varphi^i\sigma}
    =T_{t\sigma}T_{t\varphi\sigma}T_{t\varphi^2\sigma}\dots T_{t\varphi^{m-1}\sigma}
  \end{equation}
  
  definamo poi

  \begin{equation}
    t\vec{V}^m=t\varphi^m\vec V
  \end{equation}

  \begin{equation}
    t\vec{W}^m=t\vec{V}^mT^m_t
  \end{equation}

  La funzione frattale si definisce allora come

  \begin{equation}
    t\vec F=t\vec V + \sum_{i=1}^\infty t\vec W^i
  \end{equation}
\end{document}.

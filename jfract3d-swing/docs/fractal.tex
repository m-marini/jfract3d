\documentclass[a4paper,11pt]{article}
\usepackage[T1]{fontenc}
\usepackage[utf8]{inputenc}
\usepackage{lmodern}
\usepackage[italian]{babel}
\usepackage{graphicx}
\begin{document}
  \title{Spazio degli stati \thanks{Versione \$Id: Stati.tex,v 1.1 2007/08/31 01:19:50 marco Exp $ $}}
  \author{Marco Marini}
  \maketitle
  \part{Main}
  \section{Base}
  
  Sia $ t \vec{V} $ una funzione vettoriale in $ t $ con $ 0 \le t \le 1 $.
  
  Siano $ T_i $ $ n $ funzioni di trasformazione di vettori con $ i = 1 \dots n $.
  
  Sia $ t \varphi = n t - floor(nt) $ funzione di $ t $ abbiamo che $ 0 \le t \varphi < 1 $.
  
  Sia $ t \sigma = floor( n t ) $ funzione intera di $ t $ abbiamo che $ t \sigma = 0 \dots (n - 1) $.
  
  Definiamo
  
  \begin{equation}
    T^m_t = \prod_{i = 1}^{m} T_{t \varphi^{i - 1} \sigma}
    =T_{t \sigma} T_{t \varphi \sigma}T_{t \varphi^2 \sigma}\dots T_{t \varphi^{m - 1} \sigma}
  \end{equation}
  
  definamo poi

  \begin{equation}
    t \vec{V}^m = t \varphi^m \vec V
  \end{equation}

  \begin{equation}
    t \vec{W}^m = t \vec{V}^m T^m_t
  \end{equation}

  La funzione frattale si definisce allora come

  \begin{equation}
    t \vec F = t \vec V + \sum_{i = 1}^\infty t \vec W^i
  \end{equation}

\section{Piano}
Calcoliamo l'equazione del piano passante per 3 punti $ P_1 = (0, y_1, 0), P_2 = (x_2, y_2, 0) , P_3 = (0, y_3, z_3) $
\[
	\vec V = A \times \vec W
\]

\[
\left |
\begin{array}{l}
	x
\\
	y
\\
	z
\end{array}
\right |
=
\left |
\begin{array}{lll}
	a_{11} & a_{12} & a_{13}
\\
	a_{21} & a_{22} & a_{23}
\\
	a_{31} & a_{32} & a_{33}
\end{array}
\right |
\left |
\begin{array}{l}
	u
\\
	v
\\
	1
\end{array}
\right |
=
\left |
\begin{array}{l}
	a_{11} u + a_{12} v + a_{13}
\\
	a_{21} u + a_{22} v + a_{23}
\\
	a_{31} u + a_{32} v + a_{33}
\end{array}
\right |
\]

poniamo che

\[
T = 
\left |
\begin{array}{lll}
	1 & 0 & 0
\\
	a_{21} & a_{22} & a_{23}
\\
	0 & 0 & 1
\end{array}
\right |
\]

quindi

\[
 \vec V = 
\left |
\begin{array}{l}
	u
\\
	a_{21} u + a_{22} v + a_{23}
\\
	v
\end{array}
\right |
\]

\[
T^{-1} = 
\left |
\begin{array}{lll}
	1 & 0 & 0
\\
	a_{21} & a_{22} & a_{23}
\\
	0 & 0 & 1
\end{array}
\right |
\]


\end{document}